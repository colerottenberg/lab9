\documentclass{article}

\usepackage{fancyhdr}
\usepackage{graphicx}
\usepackage{float}
\usepackage{geometry}
\geometry{margin=1in}
\setlength{\headheight}{49.4pt}
\pagestyle{fancy}
\fancyhf{}
\rhead{
    \centering
  \begin{tabular*}{\textwidth}{@{\extracolsep{\fill}}ccc}
    \small University of Florida & \textbf{\small EEL3111C - Circuits} & \small Rottenberg, Cole Harrison\\
    \small Electrical \& Computer Engineering Dept. & \small Lab 9: Filters & \small Class \#: 20931 \\
    \small Page \thepage & \small Revision: 0 & \small \today \\
  \end{tabular*}
}

\begin{document}
% I need a bold horizontal line

\begin{center}
    \hrule
    \vspace{0.2cm}
    \textbf{\large REQUIREMENTS NOT MET}
    \vspace{0.2cm}
    \hrule
\end{center}
% Bullet points on what was not met
\begin{itemize}
    \item All requirements were met.
\end{itemize}

\begin{center}
    \hrule
    \vspace{0.2cm}
    \textbf{\large PROBLEMS ENCOUNTERED}
    \vspace{0.2cm}
    \hrule
\end{center}
% More Bullet points with filler text
\begin{itemize}
    \item No problems were encountered.
\end{itemize}

\begin{center}
    \hrule
    \vspace{0.2cm}
    \textbf{\large INTRODUCTION}
    \vspace{0.2cm}
    \hrule
\end{center}

In Lab 9, we explore low and high pass RC filters.
We also define differences between active and passive filters.

\begin{center}
    \hrule
    \vspace{0.2cm}
    \textbf{\large DISCUSSION}
    % A horizontal line here
    \vspace{0.2cm}
    \hrule
\end{center}

\textbf{\large 9.5 Pre-Lab Requirements:}

\textbf{9.5.1 LTSpice Simulations:}
\begin{enumerate}
    \item Review AC Analysis in LTSpice
    \item Build a simple lowpass filter, Figure 9.2a, but set R = 10 k Ohm and C =
    0.001 $\mu F$. Set the voltage source to an AC amplitude of 1 and run an AC
    analysis with the following settings: Decade, 100, 1, 1Meg. Save an image of
    the circuit, a plot of the output, and table the 3 dB frequency for submission.
    \begin{figure}[H]
        \centering
        \includegraphics[width=0.5\textwidth]{2simPlot.png}
        \caption{Plot of Low Pass Filter}
    \end{figure}
    \begin{figure}[H]
        \centering
        \includegraphics[width=0.5\textwidth]{2simCircuit.png}
        \caption{Circuit of Low Pass Filter}
    \end{figure}
    \begin{tabular}{|c|c|c|}
        \hline
        LOW-PASS & 1.6 kHz & $45\deg$ \\
        \hline
        \end{tabular}
    \item High Pass Filter
    \begin{figure}[H]
        \centering
        \includegraphics[width=0.5\textwidth]{3simPlot.png}
        \caption{Plot of High Pass Filter}
    \end{figure}
    \begin{figure}[H]
        \centering
        \includegraphics[width=0.5\textwidth]{3simCircuit.png}
        \caption{Circuit of High Pass Filter}
    \end{figure}
    \begin{tabular}{|c|c|c|}
        \hline
        HIGH-PASS & 1.063 kHz & $45\deg$ \\
        \hline
        \end{tabular}
    \item Active Low Pass Filter with $R = 1k \Omega$ and $C = 0.1 \mu F$
    \begin{figure}[H]
        \centering
        \includegraphics[width=0.5\textwidth]{4simPlot.png}
        \caption{Plot of Active Low Pass Filter}
    \end{figure}
    \begin{figure}[H]
        \centering
        \includegraphics[width=0.5\textwidth]{4simCircuit.png}
        \caption{Circuit of Active Low Pass Filter}
    \end{figure}
    \begin{tabular}{|c|c|c|}
        \hline
        ACTIVE LOW-PASS & 1.59 kHz & $45\deg$ \\
        \hline
        \end{tabular}
    \item Active High Pass Filter with $R_1 = 3.3k \Omega, R_2 = 33k \Omega$ and $C = 0.1 \mu F$
    \begin{figure}[H]
        \centering
        \includegraphics[width=0.5\textwidth]{5simPlot.png}
        \caption{Plot of Active High Pass Filter}
    \end{figure}
    \begin{figure}[H]
        \centering
        \includegraphics[width=0.5\textwidth]{5simCircuit.png}
        \caption{Circuit of Active High Pass Filter}
    \end{figure}
    \begin{tabular}{|c|c|c|}
        \hline
        ACTIVE HIGH-PASS & 482.3 Hz & $45\deg$ \\
        \hline
        \end{tabular}
\end{enumerate}
\textbf{9.5.2 Breadboard Implementation:}
\begin{enumerate}
    \item Review Network Analyzer tool in Digilent Waveforms.
    \item Build Active Low Pass Filter with $R = 1k \Omega$ and $C = 0.1 \mu F$.
    \item Network Analysis of Circuit
    \begin{figure}[H]
        \centering
        \includegraphics[width=0.5\textwidth]{3physPlot.png}
        \caption{Plot of Active Low Pass Filter}
    \end{figure}
\end{enumerate}

\textbf{\large 9.7 Write-Up:}

\begin{tabular}{|c|c|c|c|c|}
    \hline
    & Low-Pass & High-Pass & Active Low-Pass & Active High-Pass\\
    \hline
    Simulated & 16 kHz & 1.063 kHz & 1.59 kHz & 482.3 Hz \\
    \hline
    Actual & 15.674 kHz & 1.035 kHz & 1.593 kHz & 460.9 Hz \\
    \hline 
    Percent Error & 2.03\% & 2.63\% & 0.1\% & 4.44\% \\
    \hline
\end{tabular}

\begin{figure}[H]
    \centering
    \includegraphics[width=1\textwidth]{1inlab.png}
    \caption{Passive Low Pass Filter}
\end{figure}

\begin{figure}[H]
    \centering
    \includegraphics[width=1\textwidth]{3inlab.png}
    \caption{Passive High Pass Filter}
\end{figure}

\begin{figure}[H]
    \centering
    \includegraphics[width=1\textwidth]{2inlab.png}
    \caption{Active Low Pass Filter}
\end{figure}

\begin{figure}[H]
    \centering
    \includegraphics[width=1\textwidth]{4inlab.png}
    \caption{Active High Pass Filter}
\end{figure}
\begin{center}
    \hrule
    \vspace{0.2cm}
    \textbf{\large CONCLUSION}
    % A horizontal line here
    \vspace{0.2cm}
    \hrule
\end{center}

The purpose of lab 9 is to explore the working characteristics of RC filters.
However, we explore how to create active filters which can create a gain and a filter in one amplifier.
As we can see from the previous section, our physical circuits worked within a 5\% tolerance of simulated values.

\end{document}